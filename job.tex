\section{Position and duties}
The position covered during the internship period was that of software developer.
This included C and C++ on embedded platforms, and C++ and Python on Linux.

The scope and topic of the various projects was not fixed and changed several times over the course of the internship. 
In particular, three main subjects can be isolated, and will be discussed individually in the following sections:

\begin{description}  
\item [UAVCAN] Introduction on the activities and projects related to drone components supporting the UAVCAN communication standard
\item [Internal autopilot] Design and implementation of autopilot platform for internal research  
\item [Light gimbal] Development project for automotive/defense product 
\end{description}


\subsection{UAVCAN}
The initial part of the internship consisted in becoming accustomed with the tools and development platform used internally, at that time consisting of an NXP series of microcontrollers called LPC11xx.
This was accomplished with a series of programming tasks, such as interfacing the evaluation board with different sensors and components.

After having acquired the necessary understanding of the development environment, the focus shifted into experimenting with the UAVCAN library in order to evaluate its capabilities and pitfalls, before integrating it into the existing codebase.
UAVCAN is an open-source project based on CANbus aimed at simplifying communication within a drone periferals.
Once assessed and verified its functionalities, it has been chosen to start working on a line of compliant products.
The first item was the \emph{UAVCAN UAX board}, a device that could be used to control digital servos and other actuators by means of PWM signals.


\subsection{Internal autopilot}



\subsection{Light gimbal}




- uavcan
- uavcan aux board, mpu guard
- autopilot, chibios port
- wiseled