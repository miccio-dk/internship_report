\section{Position and duties}
The position covered during the internship period was that of software developer.
This included C and C++ on embedded platforms, and C++ and Python on Linux.

The scope and topic of the various projects was not fixed and changed several times over the course of the internship. 
In particular, three main subjects can be isolated, and will be discussed individually in the following sections:

\begin{description}  
\item [UAVCAN] Introduction on the activities and projects related to drone components supporting the UAVCAN communication standard
\item [Internal autopilot] Design and implementation of autopilot platform for internal research  
\item [Light gimbal] Development project for automotive/defense product 
\end{description}


\subsection{UAVCAN}
The initial part of the internship consisted in becoming accustomed with the tools and development platform used internally, at that time consisting of NXP series of microcontrollers called LPC11xx and LPC17xx.
This was accomplished with a series of programming tasks, such as interfacing the evaluation board with different sensors and components.

After having acquired the necessary understanding of the development environment, the focus shifted into experimenting with the UAVCAN library in order to evaluate its capabilities and pitfalls, before integrating it into the existing codebase.
UAVCAN is an open-source project based on CANbus aimed at simplifying communication within a drone periferals.
Once assessed and verified its functionalities, it has been chosen to start working on a line of compliant products.
The first items were the \emph{UAVCAN UAX board}, a device that can control digital servos and other actuators by means of PWM signals, the \emph{UAVCAN MPU Guard}, which automatically chooses the most efficient power source from those connected to its inputs, and the \emph{UAVCAN Debugger}, which would allow users to probe into their UAVCAN networks and perform several actions (configuring nodes, upload firwares\dots).

Programming these functionalities implied developing an Hardware Abstraction Layer (HAL) around microcontroller peripherals like ADC, DAC, timers, watchdog, and so forth.
This, in turn, required a fairly solid knowledge of the information in the data sheet and user manual of the MCU.
Another important aspect was the integration of the UAVCAN software library, which proved to be a challenging task due to the scarcity of memory resources - both storage and SRAM - in some of the devices.

Despite having gotten very close to completion on the \emph{UAVCAN UAX board}, the project has been put on hold due to major hardware design changes which made the firmware incompatible with the newer iteration.


\subsection{Internal autopilot}
At a later point, it has been agreed to develop an autopilot from stratch, in order to gain an insightful view of how estimation algorithms and navigation systems work.

My involvment in this project consisted in assisting the design of both software and hardware.
Firstly, 


\subsection{Light gimbal}




- autopilot, chibios port
- wiseled