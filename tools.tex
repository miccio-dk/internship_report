\section{Tools and knowledge used}
This section will cover various tool that have been employed during the course of the internship.
Due to the nature of the job, all of the tools presented here are software-based, although conventional Electronic Engineering tools such as power supplies and oscilloscopes have been used.

\subsection{Software and tools}

\subsubsection{LPCXpresso - LPCOpen}
LPCXpresso is NXP's own development environment.
Under that name, they market their series of evaluation boards and IDE, which is based on the popular \emph{Eclipse}. 
It features a text editor/project manager, an C/C++ compiler for the ARM architecture (based on the open-source \emph{GCC}), and several debugging tools like memory browser, peripherals and registers viewer, and firmware flasher.
The evaluation boards are build around the LPC family of 32-bit microcontrollers, and come in countless varieties (normally one or more for each MCU line).


\subsubsection{GitHub/git}
Managing large projects with several team members can easely become a tedious and time-consuming task if no version control system is used. 

In order to overcome this difficulty, \emph{git} has been chosen due to its performances, ease of use, well-maintained documentation, and the existence of online platforms like \emph{GitHub}, which allows repositories to be hosted online, shared, and managed from the website interface.

With this tool, it was possible to keep track of the different code releases, split development by means of branching a repository, and collect useful statistics to asses the state of a project.

Learning to use \emph{git} is an ongoing process, which also involves using several graphical tools.
Amongst t


\subsubsection{Prototyping platforms}
mbed/Arduino/Processing


\subsubsection{Linux}
Linux is an open-source, community-driven operating system.
Although having had previous experience with it, using it on a daily basis allowed me to become proficient with a wide range of terminal commands, like the aforementioned \emph{git}, but also \emph{grep} for searching file contents, \emph{bash} and \emph{awk} scripting, and the \emph{apt-get} software package manager.

The Linux distribution of preference was \emph{Linux Mint 17.2}, which is based on the more popular \emph{Ubuntu Linux 14.}.
It has been choses for its lightweight and features-rich desktop manager, which allowed unprecedented stability: the workstation required a single reboot throughout



\subsection{Relevant knowledge}

- control theory
- machine architecture
- basic analog